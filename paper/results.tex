\documentclass[11pt]{article}
\usepackage[utf8]{inputenc}
\usepackage{xcolor,comment, subfiles, graphicx, caption, longtable, subfig, fancyhdr}
\usepackage{float} %Used to force images to appear in the section in which it's declared
\usepackage[hidelinks]{hyperref}
\usepackage{parskip}
\usepackage[shortlabels]{enumitem}
\usepackage{multirow}
\usepackage{color}
\usepackage[toc,page]{appendix}
\usepackage{float}
\usepackage[a4paper,width=150mm,top=25mm,bottom=25mm]{geometry}
\usepackage{longtable}
\usepackage{colortbl}
\usepackage{xcolor}
\usepackage{color}
\usepackage{amsfonts}
\usepackage{amssymb}
\usepackage{amsmath}
\usepackage{cancel}

\title{DASA - Project}
\makeatletter

\begin{document}

\setlength{\parskip}{1em}

\begin{titlepage}
\centering
\vfill
{
\includegraphics[width =\linewidth, height = 4cm, keepaspectratio]{PolitecnicoLogo.png}
\label{fig:PolitecnicoLogo}
\large \\[2ex]M.Sc. Computer Science and Engineering\\
\large Data Analysis for Smart Agriculture\\[12ex]
\huge
Data Analysis On An Eggs Farm\\[1.5ex]
\large
\vspace{10mm}

\vspace{15mm}
\normalsize

\vspace{30mm}

\begin{tabular}{lclcl}
    Davide Canali & - & 10674880 & - & davide1.canali@polimi.it\\
    Matteo Cordioli & - & 00000000 & - & matteo.cordioli@polimi.it\\
    Federico Camilletti & - & 00000000 & - & federico.camilletti@polimi.it\\
    Shakiba Shahidiani  & - & 00000000 & - & shakiba.shahidiani@polimi.it\\
\end{tabular}

\vspace{30mm}

\@date\\[2.5ex]
}
\end{titlepage}

\makeatother
\tableofcontents
\newpage

\section{Introduction}
In this study, we are going to analyze data from an eggs farm near Mantova to see if it's possible to improve both animal welfare and farmer revenue.

The farm under analysis is [//TODO INSERT NAME] and has around 40'000 chickens that produce organic eggs. We have the data starting from 2014, the production of eggs is divided into cycles lasting about 13 - 15 months each. We have 5 complete cycles and the current 2022 cycle. The first 2 cycles (called X and Y) are non-organic which means the chickens are treated differently from the last 4 cycles (Z, A, B, C) which are organic.

Upon talking with the farmer we focus our attention on 3 main topics which involve:
\begin{itemize}
    \item Understanding the mortality between different cycles and organic with non-organic.
    \item Improve the welfare of the chickens.
    \item Quantify the monetary loss when a chicken dies at the start of the cycle.
\end{itemize}

\section{Analysis of Each Cycle}
\end{document}